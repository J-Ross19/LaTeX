%1st Discrete Mathematics assignment

\documentclass{article}
\usepackage{times}


\begin{document}

% top matter of the article
\title{Assignment 1: CS205}

\author{Joshua Ross\\
	RUID: 185008511\\
	Rutgers University - New Brunswick\\
	\texttt{jjr276@scarletmail.rutgers.edu}}

\date{18 September 2019}

\maketitle

\section*{1.1}
\subsubsection*{32c)}
Problem: Construct a truth table for $p \bigoplus (p \bigvee q)$
\begin{center}
\begin{tabular}{|c|c|c|c|}
\hline
$p$ & $q$ & $p \bigvee q$ & $p \bigoplus (p \bigvee q)$ \\
\hline
T & T & T & F \\
T & F & T & F \\
F & T & T & T \\
F & F & F & F \\
\hline
\end{tabular}
\end{center}
\subsubsection*{32f)}
Problem: Construct a truth table for $(p \bigoplus q) \bigwedge (p \bigoplus \neg q)$
\begin{center}
\begin{tabular}{|c|c|c|c|c|c|}
\hline
$p$ & $q$ & $\neg q$ & $p \bigoplus q$ & $p \bigoplus \neg q$ & $(p \bigoplus q) \bigwedge (p \bigoplus \neg q)$ \\
\hline
T & T & F & F & T & F\\
T & F & T & F & F & F\\
F & T & F & T & F & F\\
F & F & F & F & T & F\\
\hline
\end{tabular}
\end{center}
\subsubsection*{36c)}
Problem: Construct a truth table for $(p \bigvee q) \bigvee r$
\begin{center}
\begin{tabular}{|c|c|c|c|c|}
\hline
$p$ & $q$ & $r$ & $p \bigvee q$ & $(p \bigvee q) \bigvee r$ \\
\hline
T & T & T & T & T \\
T & T & F & T & T \\
T & F & T & T & T \\
T & F & F & T & T \\
F & T & T & T & T \\
F & T & F & T & T \\
F & F & T & F & T \\
F & F & F & F & F \\
\hline
\end{tabular}
\end{center}
\subsubsection*{36f)}
Problem: Construct a truth table for $(p \bigwedge q) \bigvee r$
\begin{center}
\begin{tabular}{|c|c|c|c|c|c|}
\hline
$p$ & $q$ & $r$ & $\neg r$  & $p \bigwedge q$ & $(p \bigwedge q) \bigvee r$ \\
\hline
T & T & T & F & T & T\\
T & T & F & T & T & T\\
T & F & T & F & F & F\\
T & F & F & T & F & T\\
F & T & T & F & F & F\\
F & T & F & T & F & T\\
F & F & T & F & F & F\\
F & F & F & T & F & T\\
\hline
\end{tabular}
\end{center}
\section*{1.3}
\subsubsection*{26)}
Problem: Use boolean algebra to show that $\neg p \rightarrow (q \rightarrow r)$ and $q \rightarrow (p \bigvee r)$ are logically equivalent

\begin{itemize}
	\item $\neg p \rightarrow (q \rightarrow r) \equiv p \bigvee (\neg q \bigvee r)$ by the implication law
	\item $p \bigvee (\neg q \bigvee r) \equiv \neg q \bigvee (p \bigvee r)$ by the commutative and associative laws
	\item $\neg q \bigvee (p \bigvee r) \equiv q \rightarrow (p \bigvee q)$ by the implication law 
\end{itemize}
\subsubsection*{30)}
Problem: Use boolean algebra to show that $(p \bigvee q) \bigwedge (\neg p \bigvee r) \rightarrow (q \bigvee r)$ is a tautology

\begin{itemize}
	\item $(p \bigvee q) \bigwedge (\neg p \bigvee r) \rightarrow (q \bigvee r) \equiv \neg ((p \bigvee q) \bigwedge (\neg p \bigvee r)) \bigvee (\neg p \bigvee r)$ by the implication law
	\item $\neg ((p \bigvee q) \bigwedge (\neg p \bigvee r)) \bigvee (\neg p \bigvee r) \equiv \neg (p \bigvee q) \bigvee \neg (\neg p \bigvee r) \bigvee (\neg p \bigvee r)$ by De Morgan law
	\item $\neg (p \bigvee q) \bigvee \neg (\neg p \bigvee r) \bigvee (\neg p \bigvee r) \equiv \neg (p \bigvee q) \bigvee T$ by the Associative and Negation law
	\item $\neg (p \bigvee q) \bigvee T \equiv T$ by Domination law, proving $(p \bigvee q) \bigwedge (\neg p \bigvee r) \rightarrow (q \bigvee r)$ is a tautology 
\end{itemize}

\section*{1.4}
\subsubsection*{36a)}
Problem: Find a counterexample (if possible) to $\forall x(x^2 \neq x)$ where the domain of x is all real numbers\\
For $x=1$, $(1)^2=(1)$, meaning $x^2 = x$, providing a counterexample to $\forall x(x^2 \neq x)$

\subsubsection*{36b)}
Problem: Find a counterexample (if possible) to $\forall x(x^2 \neq 2)$ where the domain of x is all real numbers\\
For $x=1$, $(\sqrt{2})^2=2$, meaning $x^2=2$, providing a counterexample to $\forall x(x^2 \neq 2)$

\subsubsection*{36c)}
Problem: Find a counterexample (if possible) to $\forall x(|x| > 0)$ where the domain of x is all real numbers\\
For $x=0$, $|0| = 0$, meaning $|x| = 0$, providing a counterexample to $\forall x(|x| > 0)$

\end{document}
